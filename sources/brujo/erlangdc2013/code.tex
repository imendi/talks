\documentclass[utf8,hyperref={colorlinks=true}]{beamer}
\definecolor{links}{HTML}{2A1B81}
\hypersetup{colorlinks,linkcolor=,urlcolor=links}
\mode<presentation>
\usepackage{multicol}
\usepackage{listings}
\usepackage{helvet}
\usepackage{tikz}
\usetheme{Warsaw}
\usecolortheme{whale}
\usefonttheme[onlylarge]{structuresmallcapsserif}
\usefonttheme[onlysmall]{structurebold}
\usepackage{amsthm} % pushQED, popQED

\setbeamertemplate{footline}{}
\setbeamertemplate{navigation symbols}{}

\makeatletter
\newenvironment{btHighlight}[1][]
{\begingroup\tikzset{bt@Highlight@par/.style={#1}}\begin{lrbox}{\@tempboxa}}
{\end{lrbox}\bt@HL@box[bt@Highlight@par]{\@tempboxa}\endgroup}

\newcommand\btHL[1][]{%
  \begin{btHighlight}[#1]\bgroup\aftergroup\bt@HL@endenv%
}
\def\bt@HL@endenv{%
  \end{btHighlight}%   
  \egroup
}
\newcommand{\bt@HL@box}[2][]{%
  \tikz[#1]{%
    \pgfpathrectangle{\pgfpoint{1pt}{0pt}}{\pgfpoint{\wd #2}{\ht #2}}%
    \pgfusepath{use as bounding box}%
    \node[anchor=base west, fill=orange!30,outer sep=0pt,inner xsep=1pt, inner ysep=0pt, rounded corners=3pt, minimum height=\ht\strutbox+1pt,#1]{\raisebox{1pt}{\strut}\strut\usebox{#2}};
  }%
}
\makeatother

%\usebackgroundtemplate%
{%
%    \includegraphics[width=\paperwidth,height=\paperheight]{newton.jpg}%
%}


\begin{document}

\newcommand*\oldmacro{}%
\let\oldmacro\insertshorttitle%
\renewcommand*\insertshorttitle{%
  \oldmacro\hfill%
  \insertframenumber\,/\,\inserttotalframenumber}

%%%%%%%%%%%%%%%%%%%%%%%%%%%%%%%%%%%%%%%%%%%%%%%%%%%%%%%%%%%%%%%%%%%%%%
%% CODE SNIPPETS
%%%%%%%%%%%%%%%%%%%%%%%%%%%%%%%%%%%%%%%%%%%%%%%%%%%%%%%%%%%%%%%%%%%%%%
\definecolor{darkblue}{rgb}{0,0.08,0.45} 

\lstset{% general command to set parameter(s)
		mathescape=true,
		language=erlang,
		basicstyle=\ttfamily\large,
		keywordstyle=\color{blue}\bfseries,
		identifierstyle=\color{darkblue},
		stringstyle=\ttfamily,
		moredelim=**[is][{\btHL[fill=green!30,draw=red,dashed,thin]}]{@}{@},
		showstringspaces=false}

\defverbatim[colored]\codeblocka{%
\begin{lstlisting}[]
1> lucene_server:start().
ok
\end{lstlisting}
}

\defverbatim[colored]\codeblockb{%
\begin{lstlisting}[]
2> User1 =
	[{name, "Fernando"}
	,{nick, "elbrujohalcon"}].
[{name, "Fernando"},{nick, "elbrujohalcon"}]

3> User2 =
	[{name, "Ariel Ortega"}
	,{nick, "burrito"}].
[{name, "Ariel Ortega"},{nick, "burrito"}]

4> lucene:add([User1,User2]).
ok
\end{lstlisting}
}

\defverbatim[colored]\codeblockc{%
\begin{lstlisting}[]
5> lucene:match("nick: b*", 10).
{[[{name,"Fernando Benavides"},
   {nick,"elbrujohalcon"},
   {'`score',1.0}]],
 [{total_hits,1},
  {first_hit,1},
  {query_time,783},
  {search_time,457}]}
\end{lstlisting}
}

\defverbatim[colored]\codeblockd{%
\begin{lstlisting}[basicstyle=\ttfamily\scriptsize]
6> lucene:add([[{code, X}] || X <- lists:seq(1,10)]).
ok

7> {_, M} = lucene:match("code:[0 TO 10]", 5).
{[[{code,1},{'`score',1.0}],
  [{code,2},{'`score',1.0}],
  [{code,3},{'`score',1.0}],
  [{code,4},{'`score',1.0}],
  [{code,5},{'`score',1.0}]],
 [{total_hits,10},
  {first_hit,1},
  {query_time,14335},
  {search_time,3811},
  {next_page,<<172,237,0,5,115,114,0,36,99,111,109,46,
               116,105,103,101,114,116,101,120,...>>}]}

8> lucene:continue(proplists:get_value(next_page, M), 5).
{[[{code,6},{'`score',1.0}],
  [{code,7},{'`score',1.0}],
  [{code,8},{'`score',1.0}],
  [{code,9},{'`score',1.0}],
  [{code,10},{'`score',1.0}]],
 [{total_hits,10},
  {first_hit,6},
  {query_time,2368},
  {search_time,1731}]}
\end{lstlisting}
}

\defverbatim[colored]\codeblocke{%
\begin{lstlisting}[basicstyle=\ttfamily\scriptsize]
{pre_hooks, [{compile, "mkdir -p bin"},
             {compile, "./copy-jinterface.sh"}]}.

{post_hooks,
  [{clean,
    "rm -rf bin priv/lucene_server.jar priv/OtpErlang.jar"},
   {compile,
    "javac -g -deprecation -sourcepath java_src 
           -classpath ./bin:./priv/* -d bin
           java_src/*/*/*/*.java"},
   {compile, "jar cf priv/lucene-server.jar -C bin ."}]}.
\end{lstlisting}
}

\defverbatim[colored]\codeblockf{%
\begin{lstlisting}[basicstyle=\ttfamily\scriptsize]
init([]) ->
...
  Port =
    @erlang:open_port@(
      {@spawn_executable@, Java},
      [{line,1000}, stderr_to_stdout,
       {args, JavaArgs ++
                    ["-classpath", Classpath,
                     "com.tigertext.lucene.LuceneNode",
                     ThisNode, JavaNode, erlang:get_cookie(),
                     integer_to_list(AllowedThreads)]}]),
  @wait_for_ready@(
    #state{java_port = Port, java_node = JavaNode})
end.
...
wait_for_ready(State = #state{java_port = Port}) ->
  receive
    {Port, {data, {eol, "READY"}}} ->
      true = @link(process())@,
      true = @erlang:monitor_node(State#state.java_node, true)@,
      {ok, State};
    Info ->
      ...
  end.
\end{lstlisting}
}

\lstset{% general command to set parameter(s)
		mathescape=true,
		language=java,
		basicstyle=\ttfamily\large,
		keywordstyle=\color{blue}\bfseries,
		identifierstyle=\color{darkblue},
		stringstyle=\ttfamily,
		moredelim=**[is][{\btHL[fill=green!30,draw=red,dashed,thin]}]{@}{@},
		showstringspaces=false}

\defverbatim[colored]\codeblockg{%
\begin{lstlisting}[basicstyle=\ttfamily\scriptsize]
public static void main(String[] args) {
  String peerName = args.length >= 1 ? args[0]
      : "lucene_server@localhost";
  String nodeName = args.length >= 2 ? args[1]
      : "lucene_server_java@localhost";

  try {
    NODE = args.length >= 3 ? @new OtpNode(nodeName, args[2])@
        : @new OtpNode(nodeName)@;
    PEER = peerName;

    final OtpGenServer server = @new LuceneServer(NODE)@;

    server.start();

    @System.out.println("READY")@;

  } catch (IOException e1) {
    jlog.severe("Couldn't create node: " + e1);
    e1.printStackTrace();
    System.exit(1);
  }
}
\end{lstlisting}
}

\lstset{% general command to set parameter(s)
		mathescape=true,
		language=erlang,
		basicstyle=\ttfamily\large,
		keywordstyle=\color{blue}\bfseries,
		identifierstyle=\color{darkblue},
		stringstyle=\ttfamily,
		moredelim=**[is][{\btHL[fill=green!30,draw=red,dashed,thin]}]{@}{@},
		showstringspaces=false}

%%%%%%%%%%%%%%%%%%%%%%%%%%%%%%%%%%%%%%%%%%%%%%%%%%%%%%%%%%%%%%%%%%%%%%
\begin{frame}[plain]{}{}
\only<+>{\codeblocka}
\only<+>{\codeblockb}
\only<+>{\codeblockc}
\only<+>{\codeblockd}
\only<+>{\codeblocke}
\only<+>{\codeblockf}
\only<+>{\codeblockg}
\end{frame}

\end{document}