\documentclass[utf8,hyperref={colorlinks=true}]{beamer}
\definecolor{links}{HTML}{2A1B81}
\hypersetup{colorlinks,linkcolor=,urlcolor=links}
\mode<presentation>
\usepackage{multicol}
\usepackage{listings}
\usepackage{helvet}
\usepackage{tikz}
\usetheme{Warsaw}
\usecolortheme{whale}
\usefonttheme[onlylarge]{structuresmallcapsserif}
\usefonttheme[onlysmall]{structurebold}
\usepackage{amsthm} % pushQED, popQED

\setbeamertemplate{footline}{}
\setbeamertemplate{navigation symbols}{}

\makeatletter
\newenvironment{btHighlight}[1][]
{\begingroup\tikzset{bt@Highlight@par/.style={#1}}\begin{lrbox}{\@tempboxa}}
{\end{lrbox}\bt@HL@box[bt@Highlight@par]{\@tempboxa}\endgroup}

\newcommand\btHL[1][]{%
  \begin{btHighlight}[#1]\bgroup\aftergroup\bt@HL@endenv%
}
\def\bt@HL@endenv{%
  \end{btHighlight}%   
  \egroup
}
\newcommand{\bt@HL@box}[2][]{%
  \tikz[#1]{%
    \pgfpathrectangle{\pgfpoint{1pt}{0pt}}{\pgfpoint{\wd #2}{\ht #2}}%
    \pgfusepath{use as bounding box}%
    \node[anchor=base west, fill=orange!30,outer sep=0pt,inner xsep=1pt, inner ysep=0pt, rounded corners=3pt, minimum height=\ht\strutbox+1pt,#1]{\raisebox{1pt}{\strut}\strut\usebox{#2}};
  }%
}
\makeatother

%\usebackgroundtemplate%
{%
%    \includegraphics[width=\paperwidth,height=\paperheight]{newton.jpg}%
%}


\begin{document}

\newcommand*\oldmacro{}%
\let\oldmacro\insertshorttitle%
\renewcommand*\insertshorttitle{%
  \oldmacro\hfill%
  \insertframenumber\,/\,\inserttotalframenumber}

%%%%%%%%%%%%%%%%%%%%%%%%%%%%%%%%%%%%%%%%%%%%%%%%%%%%%%%%%%%%%%%%%%%%%%
%% CODE SNIPPETS
%%%%%%%%%%%%%%%%%%%%%%%%%%%%%%%%%%%%%%%%%%%%%%%%%%%%%%%%%%%%%%%%%%%%%%
\definecolor{darkblue}{rgb}{0,0.08,0.45} 

\lstset{% general command to set parameter(s)
		mathescape=true,
		language=erlang,
		basicstyle=\ttfamily\large,
		keywordstyle=\color{blue}\bfseries,
		identifierstyle=\color{darkblue},
		stringstyle=\ttfamily,
		moredelim=**[is][{\btHL[fill=green!30,draw=red,dashed,thin]}]{@}{@},
		showstringspaces=false}

\defverbatim[colored]\codeblocka{%
\begin{lstlisting}[]
1> lucene_server:start().
ok
\end{lstlisting}
}

\defverbatim[colored]\codeblockb{%
\begin{lstlisting}[]
2> User1 =
	[{name, "Fernando"}
	,{nick, "elbrujohalcon"}].
[{name, "Fernando"},{nick, "elbrujohalcon"}]

3> User2 =
	[{name, "Ariel Ortega"}
	,{nick, "burrito"}].
[{name, "Ariel Ortega"},{nick, "burrito"}]

4> lucene:add([User1,User2]).
ok
\end{lstlisting}
}

\defverbatim[colored]\codeblockc{%
\begin{lstlisting}[]
5> lucene:match("nick: b*", 10).
{[[{name,"Fernando Benavides"},
   {nick,"elbrujohalcon"},
   {'`score',1.0}]],
 [{total_hits,1},
  {first_hit,1},
  {query_time,783},
  {search_time,457}]}
\end{lstlisting}
}

\defverbatim[colored]\codeblockd{%
\begin{lstlisting}[basicstyle=\ttfamily\scriptsize]
6> lucene:add([[{code, X}] || X <- lists:seq(1,10)]).
ok

7> {_, M} = lucene:match("code:[0 TO 10]", 5).
{[[{code,1},{'`score',1.0}],
  [{code,2},{'`score',1.0}],
  [{code,3},{'`score',1.0}],
  [{code,4},{'`score',1.0}],
  [{code,5},{'`score',1.0}]],
 [{total_hits,10},
  {first_hit,1},
  {query_time,14335},
  {search_time,3811},
  {next_page,<<172,237,0,5,115,114,0,36,99,111,109,46,
               116,105,103,101,114,116,101,120,...>>}]}

8> lucene:continue(proplists:get_value(next_page, M), 5).
{[[{code,6},{'`score',1.0}],
  [{code,7},{'`score',1.0}],
  [{code,8},{'`score',1.0}],
  [{code,9},{'`score',1.0}],
  [{code,10},{'`score',1.0}]],
 [{total_hits,10},
  {first_hit,6},
  {query_time,2368},
  {search_time,1731}]}
\end{lstlisting}
}

\defverbatim[colored]\codeblocke{%
\begin{lstlisting}[basicstyle=\ttfamily\scriptsize]
{pre_hooks, [{compile, "mkdir -p bin"},
             {compile, "./copy-jinterface.sh"}]}.

{post_hooks,
  [{clean,
    "rm -rf bin priv/lucene_server.jar priv/OtpErlang.jar"},
   {compile,
    "javac -g -deprecation -sourcepath java_src 
           -classpath ./bin:./priv/* -d bin
           java_src/*/*/*/*.java"},
   {compile, "jar cf priv/lucene-server.jar -C bin ."}]}.
\end{lstlisting}
}

\defverbatim[colored]\codeblockf{%
\begin{lstlisting}[basicstyle=\ttfamily\scriptsize]
init([]) ->
...
  Port =
    @erlang:open_port@(
      {@spawn_executable@, Java},
      [{line,1000}, stderr_to_stdout,
       {args, JavaArgs ++
                    ["-classpath", Classpath,
                     "com.tigertext.lucene.LuceneNode",
                     ThisNode, JavaNode, erlang:get_cookie(),
                     integer_to_list(AllowedThreads)]}]),
  @wait_for_ready@(
    #state{java_port = Port, java_node = JavaNode})
end.
...
wait_for_ready(State = #state{java_port = Port}) ->
  receive
    {Port, {data, {eol, "READY"}}} ->
      true = @link(process())@,
      true = @erlang:monitor_node(State#state.java_node, true)@,
      {ok, State};
    Info ->
      ...
  end.
\end{lstlisting}
}

\lstset{% general command to set parameter(s)
		mathescape=true,
		language=java,
		basicstyle=\ttfamily\large,
		keywordstyle=\color{blue}\bfseries,
		identifierstyle=\color{darkblue},
		stringstyle=\ttfamily,
		moredelim=**[is][{\btHL[fill=green!30,draw=red,dashed,thin]}]{@}{@},
		showstringspaces=false}

\defverbatim[colored]\codeblockg{%
\begin{lstlisting}[basicstyle=\ttfamily\scriptsize]
public static void main(String[] args) {
  String peerName = args.length >= 1 ? args[0]
      : "lucene_server@localhost";
  String nodeName = args.length >= 2 ? args[1]
      : "lucene_server_java@localhost";

  try {
    NODE = args.length >= 3 ? @new OtpNode(nodeName, args[2])@
        : @new OtpNode(nodeName)@;
    PEER = peerName;

    final OtpGenServer server = @new LuceneServer(NODE)@;

    server.start();

    @System.out.println("READY")@;

  } catch (IOException e1) {
    jlog.severe("Couldn't create node: " + e1);
    e1.printStackTrace();
    System.exit(1);
  }
}
\end{lstlisting}
}

\defverbatim[colored]\codeblocki{%
\begin{lstlisting}[basicstyle=\ttfamily\scriptsize]
public abstract class OtpGenServer extends OtpSysProcess {
  protected OtpGenServer(OtpNode host) {
    super(host);
  }

  protected OtpGenServer(OtpNode host, String name) {
    super(host, name);
  }
...
  protected abstract OtpErlangObject @handleCall@(
      OtpErlangObject cmd, OtpErlangTuple from)
      throws OtpStopException, OtpContinueException,
      OtpErlangException;

  protected abstract void @handleCast@(OtpErlangObject cmd)
      throws OtpStopException, OtpErlangException;

  protected abstract void @handleInfo@(OtpErlangObject cmd)
      throws OtpStopException;

  protected abstract void @terminate@(OtpErlangException oee);

}
\end{lstlisting}
}

\defverbatim[colored]\codeblockj{%
\begin{lstlisting}[basicstyle=\ttfamily\scriptsize]
protected OtpErlangObject handleCall(OtpErlangObject cmd,
    OtpErlangTuple from)
    throws OtpStopException, OtpContinueException,
    OtpErlangException {
    
  OtpErlangTuple cmdTuple = (OtpErlangTuple) cmd;
  OtpErlangAtom cmdName = (OtpErlangAtom) cmdTuple.elementAt(0);
  
  if (cmdName.atomValue().equals("pid")) {
    return super.getSelf();
  } else if (cmdName.atomValue().equals("match")) {
    String queryString =
      ((OtpErlangString) cmdTuple.elementAt(1)).stringValue();
    int pageSize =
      ((OtpErlangLong) cmdTuple.elementAt(2)).intValue();

    runMatch(queryString, pageSize, from);
    
    throw new OtpContinueException();

  } else
...
}
\end{lstlisting}
}

\defverbatim[colored]\codeblockm{%
\begin{lstlisting}[basicstyle=\ttfamily\scriptsize]
public LuceneServer(OtpNode host, int allowedThreads)
    throws CorruptIndexException, LockObtainFailedException,
            IOException {
  super(host, "lucene_server");
...
  Extensions ext = @new Extensions('.')@;
  ext.add("near", new NearParserExtension());
  ext.add("erlang", new ErlangParserExtension(this.translator));
  this.extensions = ext;
...
}

private QueryParser queryParser() {
  return new LuceneQueryParser(Version.LUCENE_36, this.analyzer,
      this.translator, @this.extensions@);
}
\end{lstlisting}
}

\defverbatim[colored]\codeblockn{%
\begin{lstlisting}[basicstyle=\ttfamily\scriptsize]
public Query parse(ExtensionQuery extQuery)
  throws ParseException {
  String key = extQuery.getField();
  String[] modFun = @extQuery.getRawQueryString()@.split(":");
  if (modFun.length < 2) {
    throw new ParseException(
        "erlang queries expect values in <mod>:<fun> or
         <mod>:<fun>:<args> format");
  } else {
    String mod = modFun[0], fun = modFun[1], args;
    if (modFun.length == 2) {
      args = "[]";
    } else if (modFun.length == 3) {
      args = modFun[2];
    } else {
...
    }
    ErlangFilter filter = @new ErlangFilter(mod, fun, args, key,@
        @this.translator.getFieldType(key));@
    ValueSource valSrc = new ErlangValueSource(filter);
    return new CustomScoreQuery(new ConstantScoreQuery(filter),
        new ValueSourceQuery(valSrc));
  }
}
\end{lstlisting}
}

\defverbatim[colored]\codeblocko{%
\begin{lstlisting}[basicstyle=\ttfamily\scriptsize]
public DocIdSet getDocIdSet(IndexReader reader)
  throws IOException {
  final int docBase = this.nextDocBase;
  this.nextDocBase += reader.maxDoc();
  final OtpErlangObject[] docValues;
...
  final FixedBitSet bits = new FixedBitSet(reader.maxDoc());
  OtpErlangTuple call =
    new OtpErlangTuple(new OtpErlangObject[] {
      this.mod,this.fun,this.arg, new OtpErlangList(docValues)});
...
  OtpErlangObject response = OtpGenServer.call(LuceneNode.NODE,
      "lucene", LuceneNode.PEER, call);
...
  OtpErlangList results = (OtpErlangList) response;
  for (int docid = 0; docid < docValues.length; docid++) {
    OtpErlangObject result = results.elementAt(docid);
    if (result instanceof OtpErlangDouble) {
      scores.put(docid + docBase,
          ((OtpErlangDouble) result).doubleValue());
      bits.set(docid);
    } else {
      bits.clear(docid);
    }
...
}
\end{lstlisting}
}

\lstset{% general command to set parameter(s)
		mathescape=true,
		language=erlang,
		basicstyle=\ttfamily\large,
		keywordstyle=\color{blue}\bfseries,
		identifierstyle=\color{darkblue},
		stringstyle=\ttfamily,
		moredelim=**[is][{\btHL[fill=green!30,draw=red,dashed,thin]}]{@}{@},
		showstringspaces=false}


\defverbatim[colored]\codeblockh{%
\begin{lstlisting}[basicstyle=\ttfamily\scriptsize]
do_call(Process, Label, Request, Timeout) ->
    try erlang:monitor(process, Process) of
        Mref ->
...
    catch
        error:_ ->
            %% Node (C/Java?) is not supporting the monitor.
            %% The other possible case -- this node is not 
            %% distributed -- should have been handled earlier.
            %% Do the best possible with monitor_node/2.
            %% This code may hang indefinitely if the Process 
            %% does not exist. It is only used for featureweak
            %% remote nodes.
            Node = get_node(Process),
            monitor_node(Node, true),
...
    end.
\end{lstlisting}
}

\defverbatim[colored]\codeblockk{%
\begin{lstlisting}[basicstyle=\ttfamily\scriptsize]
9> lucene:add(
    [[{index, I},{location, @lucene_utils:geo(I/2,I+1.5)@}]
     || I <- lists:seq(-10,10)]).
ok

10> lucene:match(@"location.near:0.0,1.0,100"@, 3).
{[[{index,0},
   {location,{geo,-8.381903171539307e-8,1.5000000782310963}},
   {'`score',-17.292743682861328}],
  [{index,-1},
   {location,{geo,-0.5000000540167093,0.500000137835741}},
   {'`score',-24.45547866821289}]],
 [{total_hits,2},
  {first_hit,1},
  {query_time,1984},
  {search_time,1021}]}
\end{lstlisting}
}

\defverbatim[colored]\codeblockl{%
\begin{lstlisting}[basicstyle=\ttfamily\scriptsize]
11> lucene:match(
      @"index.erlang:\"lists:map:[fun(X) -> X*1.0 end]\""@, 3).
{[[{index,10},
   {location,{geo,4.999999953433871,11.500000152736902}},
   {'`score',5.0}],
  [{index,9},
   {location,{geo,4.499999983236194,10.49999987706542}},
   {'`score',4.499999523162842}],
  [{index,8},
   {location,{geo,4.000000013038516,9.499999936670065}},
   {'`score',3.999999761581421}]],
 [{total_hits,21},
  {first_hit,1},
  {query_time,2256},
  {search_time,1240},
  {next_page,<<172,237,0,5,115,114,0,36,99,111,109,46,
               116,105,103,101,114,116,101,120,...>>}]}
\end{lstlisting}
}

\defverbatim[colored]\codeblockp{%
\begin{lstlisting}[basicstyle=\ttfamily\scriptsize]
Reply =
  try
    {ok, Scanned, _} = erl_scan:string(Args++"."),
    {ok, Parsed} = erl_parse:parse_exprs(Scanned),
    {value, Arguments, _} = erl_eval:exprs(Parsed, []),
    erlang:apply(
      Mod,Fun,
      Arguments ++ [[parse_value(Value) || Value <- Values]])
  catch
    _:Error ->
      {error, Error}
  end,
gen_server:reply(From, Reply)
\end{lstlisting}
}

\defverbatim[colored]\codeblockq{%
\begin{lstlisting}[basicstyle=\ttfamily\scriptsize]
1> erlang:monitor(
	process, {lucene_server, 'other_node@host.local'}).
#Ref<0.0.0.210>
2> erlang:monitor(
	process,
	{lucene_server, 'lucene_server_java@host.local'}).
** exception error: bad argument
     in function  monitor/2
        called as monitor(
           process,
           {lucene_server,'lucene_server_java@host.local'})
     in call from erlang:dmonitor_p/2
\end{lstlisting}
}

\defverbatim[colored]\codeblockr{%
\begin{lstlisting}[basicstyle=\ttfamily\scriptsize]
1> Pid = rpc:call(
    'other_node@host.local', erlang, whereis, [lucene_server]).
<10086.75.0>
2> link(Pid).
true
3> Pid2 = rpc:call(
    'lucene_server_java@priscilla.local', erlang,
    whereis, [lucene_server], 5000).
{badrpc,timeout}
\end{lstlisting}
}

\defverbatim[colored]\codeblocks{%
\begin{lstlisting}[basicstyle=\ttfamily\scriptsize]
1> Pid = gen_server:call(
	{lucene_server, 'lucene_server_java@priscilla.local'},
	{pid}).
<10087.1.0>
2> link(Pid).
true
\end{lstlisting}
}

%%%%%%%%%%%%%%%%%%%%%%%%%%%%%%%%%%%%%%%%%%%%%%%%%%%%%%%%%%%%%%%%%%%%%%
\begin{frame}[plain]{}{}
\only<+>{\codeblocka}
\only<+>{\codeblockb}
\only<+>{\codeblockc}
\only<+>{\codeblockd}
\only<+>{\codeblocke}
\only<+>{\codeblockf}
\only<+>{\codeblockg}
\only<+>{\codeblockh}
\only<+>{\codeblocki}
\only<+>{\codeblockj}
\only<+>{\codeblockk}
\only<+>{\codeblockl}
\only<+>{\codeblockm}
\only<+>{\codeblockn}
\only<+>{\codeblocko}
\only<+>{\codeblockp}
\only<+>{\codeblockq}
\only<+>{\codeblockr}
\only<+>{\codeblocks}
\end{frame}

\end{document}